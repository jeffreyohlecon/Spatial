\documentclass{article}
\usepackage[utf8]{inputenc}
\usepackage{amsmath}
\usepackage{graphicx}
\usepackage{xcolor}
\usepackage{hyperref}
\usepackage{soul}
\usepackage{natbib}
\bibliographystyle{plainnat}
\title{to-dos for our Spatial Paper}
\author{Your Name}
\date{\today}

\begin{document}

\maketitle
\begin{enumerate}
    \item Download all data, assuring it's in consistent years (2020 census, probably 2022 for everything else?). See Esteban's slide 14. We need CFS data, ACS data on commuting, Census Data on County Centroids, and the BEA stuff for wages and commute flows (writeup.pdf also has much more detail on this).
    
    \begin{itemize}
        \item @Henrique - we were discussing the CFS stuff. Seems you do need the CFS data for the model, to allocate deficits, not just to validate it, 
    \end{itemize}
    \item \textbf{not actually needed until counterfactual step,} In addition to the Esteban stuff we want college education (\%) by county as a proxy for the high-skill / low skill think. Which is in the ACS. 
    \item Data cleaning: Allocate deficits $D_i$, to counties from the CFS regions, as they do in the paper
    \item \st{Expressing each of these quantities etc at the CFS level instead of the county-level}
    \item Data cleaning: Compute $\bar v_i$ as in the paper 
    
    \item Should have the following from the data or very simply computed from it: $w_i$, $L_i$, $\bar v_i$
    \item Estimation: Estimate $\psi$ as they do in the paper (appendix B.5.)
    \item Assume $\sigma = 4$ as in paper
    \item Assume $1-\alpha = 0.4$ as in paper
    \item \textbf{done} Compute $d_{ni}$ given distances, $\psi$ and $\sigma$
    \item \textbf{done} Compute $A_i$ (system of $N = \approx 3000$ equations) as in (16) in the paper. Might require some clever computation step to make it not take a super long time.
    \item \textbf{up to here is not terribly bad}Estimate $\phi$ and $\epsilon$ as in appendix. 
    \item \textbf{We might not have to do this and can use exact hat} Given $\lambda_{ni}$ $A_i$, $w_i$, $L_i$, $\bar v_i$, $R_n$, $\sigma$, $\alpha$ $\phi$, $\psi$, solve for the $3000x3000$ matrix $\mathcal{B}_{ni}$, as in (17) -- THIS WILL PROBABLY BE HARD BECAUSE IT REQUIRES SOLVING FOR 9 MILLION THINGS. SEE IF WE CAN DO A CONTRACTION OR IF THERE ARE HINTS IN THE APPENDIX
    \item \textbf{done} Determine counterfacutal values for $B_{ni}$, using Jeffrey's latex writeup
    \item Run the counterfactual using exact hat algebra (Appendix B2 gives pseudocode)  NOTE! Our counterfactual can be expressed as just reducing $B_{ni}$. everything else will be fixed
    \item This uses tattonement, so might take a long amount of computer time. ( I guess we're solving for 18 million changes) -- THIS WILL ALSO PROBABLY TAKE A LONG TIME
    \item  Report results in compelling ways from counterfactual. Probably tables, maps (using county shape files). 
    \item Writing: The first 3 steps can be done in parallel with much of the above.
    \begin{itemize}
        \item Abstract and introduction (can build off Jeffrey's writeup for esteban)
        \item Background and diagnostic. This should explain how this is an intermediate step to the more ambitious model Esteban proposed with two types, and low-types wages potentially depending negatively on commute costs.  (if commute costs are high, lots of taxi drivers still get to work). He also seemed to want to show differential welfare effects by city, which is not possible in this model (expected utility equalized everywhere).
    \item Model / Theory - much of this will just be citing Monte et al
    \item Data   - much of this will just be citing Monte et al
    \item Counterfactual exercise - answer all the questions posed in the problem set and comment on our results. This will be more involved, probably. Note the PSET does say ``If you are interested in a non-local policy (i.e. a policy like a construction of interstate highways that affect many counties similtaneously), you do not need to compute changes in fundamentals with high level of precision (e.g., changes in commuting costs in each county), a rough approximation will suffice''. So we don't need to be so defensive about how we estiamte the change in commute costs.
    \end{itemize}
\end{enumerate}


\section{Notes from Jan 26 Sunday morning chat}
Before Friday 1/31: 
Yulia: up to and including 11. (can skip 2) by end of day Friday. Let us know if you need help
Jeffrey / Henrique: looking at modifying the model to drop trade entirely \textbf{Jeffrey - done} 
Jeffrey: Start on non-mathematical writing. (before Wednesday , maybe split with Henrique) \textbf{Jeffrey - added some stuff}
 
After Friday:
Henrique and Jeffrey finish counterfactual over next weekend. 


Henrique and Jeffrey pretty free to work on stuff on Tuesday and Wednesday Feb 4 / 5.
Ideal goal is to be done with everything by Wednesday Feb 5. 


Other simplifying idea

\begin{enumerate}
\item Just do one state? and solve it for each state. 
\item Or just do one state with counties.
\item Or do CFS regions
\end{enumerate}

To Dos if we do this
\begin{enumerate}
\item If we have people commuting / doing trade across states?  Need to kill / reallocate cross-state commutes (use same procedure as he does for $>120km$ commutes)
\item remove and rescale? How's this work exactly? 

\end{enumerate}

\end{document}