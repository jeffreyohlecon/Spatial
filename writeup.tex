\documentclass{article}
\usepackage[utf8]{inputenc}
\usepackage{amsmath}
\usepackage{graphicx}

\title{Our Spatial Paper}
\author{Your Name}
\date{\today}

\begin{document}

\maketitle

\begin{abstract}
    Please be concise.
\end{abstract}

\section{Introduction}
\label{sec:intro}
Your introduction here.

\section{Background and Diagnostic}
Why is the question or policy you want to examine
important? Why is a quantitative spatial model the right tool to answer the question?
Are there any particular features of the policy/economic environment that are impor-
tant for the analysis? Feel free to focus on one or a few aspects that you want to study
in depth and identify the underlying mechanisms, which will inform you about the key
elements to be included in your model.
\section{Model}

Write down the model that directly speaks to the mechanisms you identified
in the previous step. You need to specify all the building blocks and solve for the
equilibrium (equilibria). Please be clear and concise, and feel free to leave tedious
derivations in the appendix or cite any relevant derivations you want to lift from
existing papers.
You are encouraged to tweak or simplify an existing model, like the one in Monte et
al. (2018). If you want to be more creative, feel free to select components from the
“menu of quantitative spatial models” in Section 2 of Redding and Rossi-Hansberg
(2017). When tweaking an existing model or building a model of your own, make
sure you include only the necessary elements related to your proposed mechanisms.
Importantly, please be aware of the time and feasibility constraints when specifying
your model.

\section{Data and Estimation}
\label{sec:data}
Describe the data you use to estimate the model. What
variables do you need? How do you access them? How do you plan to estimate the
model? For parameters that you will calibrate, justify your choices. For parameters
that you will estimate, explain your strategy.

\section{Counterfactual}
\label{sec:counterfactual}
Carefully describe the counterfac-
tual exercise (i.e., a policy that you want to evaluate) you want to examine. What is your plan for implementing it? Are you going to fully invert the model to back out
fundamentals? Are you going to use exact hat algebra to solve the model in changes?
For the purpose of this problem set, you can restrict your attention to policies that
are local in nature (i.e., policies that target a specific county or a set of counties
independently of others). If you are interested in a non-local policy (i.e. a policy like a
construction of interstate highways that affect many counties similtaneously), you do
not need to compute changes in fundamentals with high level of precision (e.g., changes
in commuting costs in each county), a rough approximation will suffice

\section{Appendix}
Painful proofs go here.

\end{document}