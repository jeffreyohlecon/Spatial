\documentclass{article}
\usepackage[utf8]{inputenc}
\usepackage{amsmath}
\usepackage{graphicx}
\usepackage{xcolor}
\usepackage{hyperref}
\usepackage{natbib}
\bibliographystyle{plainnat}
\title{to-dos for our Spatial Paper}
\author{Your Name}
\date{\today}

\begin{document}

\maketitle
\begin{enumerate}
    \item Download all data, assuring it's in consistent years (2020 census, probably 2022 for everything else?). See Esteban's slide 14. We need CFS data, ACS data on commuting, Census Data on County Centroids, and the BEA stuff for wages and commute flows (writeup.pdf also has much more detail on this).
    
    \begin{itemize}
        \item @Henrique - we were discussing the CFS stuff. Seems you do need the CFS data for the model, to allocate deficits, not just to validate it, 
    \end{itemize}
    \item In addition to the Esteban stuff we want college education (\%) by county as a proxy for the high-skill / low skill think. Which is in the ACS. 
    \item Data cleaning: Allocate deficits $D_i$, to counties from the CFS regions, as they do in the paper
    \item Data cleaning: Compute $\bar v_i$ as in the paper 
    \item Should have the following from the data or very simply computed from it: $w_i$, $L_i$, $\bar v_i$
    \item Estimation: Estimate $\psi$ as they do in the paper (appendix B.5.)
    \item Assume $\sigma = 4$ as in paper
    \item Assume $1-\alpha = 0.4$ as in paper
    \item Compute $d_{ni}$ given distances, $\psi$ and $\sigma$
    \item Compute $A_i$ (system of $N = \approx 3000$ equations) as in (16) in the paper. Might require some clever computation step to make it not take a super long time.
    \item Estimate $\phi$ and $\epsilon$ as in appendix. 
    \item Given $\lambda_{ni}$ $A_i$, $w_i$, $L_i$, $\bar v_i$, $R_n$, $\sigma$, $\alpha$ $\phi$, $\psi$, solve for the $3000x3000$ matrix $\mathcal{B}_{ni}$, as in (17) -- THIS WILL PROBABLY BE HARD BECAUSE IT REQUIRES SOLVING FOR 9 MILLION THINGS. SEE IF WE CAN DO A CONTRACTION OR IF THERE ARE HINTS IN THE APPENDIX
    \item Determine counterfacutal values for $B_{ni}$, using Jeffrey's latex writeup
    \item Run the counterfactual using exact hat algebra (Appendix B2 gives pseudocode)  NOTE! Our counterfactual can be expressed as just reducing $B_{ni}$. everything else will be fixed
    \item This uses tattonement, so might take a long amount of computer time. ( I guess we're solving for 18 million changes) -- THIS WILL ALSO PROBABLY TAKE A LONG TIME
    \item  Report results in compelling ways from counterfactual. Probably tables, maps (using county shape files). 
    \item Writing: The first 3 steps can be done in parallel with much of the above.
    \begin{itemize}
        \item Abstract and introduction (can build off Jeffrey's writeup for esteban)
        \item Background and diagnostic.
    \item Model / Theory - much of this will just be citing Monte et al
    \item Data   - much of this will just be citing Monte et al
    \item Counterfactual exercise - answer all the questions posed in the problem set and comment on our results. This will be more involved, probably. Note the PSET does say ``If you are interested in a non-local policy (i.e. a policy like a construction of interstate highways that affect many counties similtaneously), you do not need to compute changes in fundamentals with high level of precision (e.g., changes in commuting costs in each county), a rough approximation will suffice''. So we don't need to be so defensive about how we estiamte the change in commute costs.
    \end{itemize}
\end{enumerate}



\end{document}