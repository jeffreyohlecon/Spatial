\documentclass{article}
\usepackage[utf8]{inputenc}
\usepackage{amsmath}
\usepackage{graphicx}
\usepackage{xcolor}
\usepackage{hyperref}
\usepackage{natbib}
\bibliographystyle{plainnat}
\title{Our Spatial Paper}
\author{Your Name}
\date{\today}

\begin{document}

\maketitle

\begin{abstract}
    Please be concise.
\end{abstract}

\section{Project Plan}
\begin{itemize}
    \item Idea Formation and Lit Review  Jan 13-17  [together ]
    \item Theory - Coming up with a Model/Try to meet with Esteban to get feedback? Jan 20-22 [together]
    \item Theory - Solving the Model Jan 23-24 [Key Person = Henrique]
    \item Coding (data, estimation, counterfactual, and results) - Jan 27 - Jan 31  [Key Person = ]
    \begin{itemize}
        \item Inputs are fairly divisible
        \item Outputs are less divisible
        \item Writing -  Feb 4 - 7  (KeyPerson)
   \end{itemize}
\end{itemize}
\section{Meeting Notes}
\begin{enumerate}
    \item  Could have teams of (2) for various things. 
    \item There are some tasks that are independent. 
    \item Decide to break stuff up once we actually have an idea.
    \item Henrique: Likes Lit Review / Idea Formation / Likes Writing [let us simplify model if possible]. Prefers the equilibrium part of coding to the early / data cleaning part Less good at Geodata.  Very good at polishing and visual details. 
    \item Jeffrey: Likes Coding / Likes Writing
    \item Yulia: Interested in counterfactuals part of coding. Curious about theory. Experience about writing / likes writing. 
\end{enumerate}

\subsection{ for next meeting, Jan 17 }
\begin{enumerate}
    \item Read model / counterfactuals of Monte et al in detail enough 
    \item In the Esteban choose your model 
    \item Map Monte et al to Esteban's choose your model. 
    \item Look at Jeffrey's ideas. 
\end{enumerate}



\subsection{During next meeting}
\begin{enumerate}
    \item pick idea
    \item assign data 
\end{enumerate}


\section{Introduction}
\textit{What is the question or policy you want to examine? Although we encourage
creativity, our main goal is to make you more familiar with quantitative spatial models
and give you some practice. If you have hard time coming up with an implementable
idea, you can choose from the following list of topics}




\section{Background and Diagnostic}
Prompt:Why is the question or policy you want to examine
important? Why is a quantitative spatial model the right tool to answer the question?
Are there any particular features of the policy/economic environment that are impor-
tant for the analysis? Feel free to focus on one or a few aspects that you want to study
in depth and identify the underlying mechanisms, which will inform you about the key
elements to be included in your model.

\section{Model}


\subsubsection{Simplifications}

In this draft, we ....

But in a future draft, we may consider XYZ


\subsubsection{Taking the model to the data}


\section{Data and Estimation}





\section{Counterfactual}

\section{Appendix}
Painful proofs go here.

\bibliography{references}

\end{document}