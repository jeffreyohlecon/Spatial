\documentclass{article}
\usepackage[utf8]{inputenc}
\usepackage{amsmath}
\usepackage{graphicx}
\usepackage{xcolor}
\usepackage{hyperref}
\usepackage{natbib}
\bibliographystyle{plainnat}
\title{Our Spatial Paper}
\author{Your Name}
\date{\today}

\begin{document}

\maketitle

\begin{abstract}
    Please be concise.
\end{abstract}





\section{Introduction}
\textit{What is the question or policy you want to examine? Although we encourage
creativity, our main goal is to make you more familiar with quantitative spatial models
and give you some practice. If you have hard time coming up with an implementable
idea, you can choose from the following list of topics}


\textcolor{red}{paste in jeffrey's motivation from erh meeting}
We are going to examine the impact of the introduction of autonomous vehicles (AVs) on the spatial distribution of economic activity, and welfare.

While there is substantial uncertainty as to where AVs will be most adopted, and their costs, states are currently making decisions about how to regulate AVs (citation XYZ); this exercise will be useful to those deciding how to form regulations.




\section{Background and Diagnostic}
\textit{Prompt:Why is the question or policy you want to examine
important? Why is a quantitative spatial model the right tool to answer the question?
Are there any particular features of the policy/economic environment that are impor-
tant for the analysis? Feel free to focus on one or a few aspects that you want to study
in depth and identify the underlying mechanisms, which will inform you about the key
elements to be included in your model.}



Once widely adopted, the first-order effect of AVs is a decline in commute costs. However, when we view AVs as not only a change in commute costs, but as a substitute for certain types of labor (e.g. taxi drivers, truck drivers),
 
\textcolor{red}{not sure about this 2x2 cost of housing near work or near home?}

The gain from substituting to AVs should be especially large where rents are high, since there are places where (i) wages must be high for low-skill workers, (ii) commuting  is relatively more attractive. 


We thus note this technology will have varied impacts (i) on low-skill and high-skill agents, and in (ii) areas where the cost of housing is relatively high vs. low.
This intuition is well-illustrated by the following 2x2 table: \textcolor{red}{does it need to be 2x2 or would 1x2 be enough}
\begin{table}[h!]
\centering
\begin{tabular}{|c|c|c|}
\hline
 & High Skill & Low Skill \\ \hline
High Rent & Unambig. Benefit, Declining Commute Costs & Ambig.:  Wage decline (-), but declining commute costs (+)   \\ \hline
Low Rent & Unambig. Benefit, Declining Commute Costs (but smaller than for high rent) & Ambig.:  Wage decline (-), but declining commute costs (+)   \\ \hline
\end{tabular}
\caption{Impact of AVs on different skill levels and places}
\label{tab:impact_avs}
\end{table}


The key question we hope to begin answering is \textbf{how varied benefits, across skill-levels and space, will determine the winners and losers of this technology, and the spatial distribution of economic activity}.

For tractability, we will start by capturing this intuition of the 2x2 table in a reduced-form way,  \textbf{by reducing bilateral commute costs proportionally to the population-weighted mean college-educated share of the pair. }


\section{Model}

We begin with the model of \citep{monte_commuting_2018}.




\subsubsection{Simplifications and Tweaks}

In a future draft, we may consider the following simplifications: 

\begin{enumerate}
\item removing trade (insert Jeffrey's theory stuff here)
\end{enumerate}

In a future draft, we may consider the following extensions: 
\begin{enumerate}
\item  (i) heterogeneity: explicitly model high and low skill types
 \item (ii) endogenize low-types' wages.
\item (iii) consider the forward-looking decision of a household to purchase an A.V.
\end{enumerate}



\section{Data and Estimation}
We use data on wages (based on place of work), commuting flows, distances, etc.



Estimating $\psi$ \\
Estimating $\phi$ \\ 
Estimating $\epsilon$ \\
Calculating productivity vector $A_i$ \\


We do not need to invert the bilateral amenities matrix $\mathcal{B}_{in}$. Instead we can use exact hat algebra for our counterfactuals of interest. 




\section{Counterfactual}

We take as our base economy the one with no AV adoption, and commuting costs are calibrated using existing flows.

We then shock the economy by introducing varied adoption, based on the aggregate skill-level of the county pair.

\textcolor{red}{Justify what we do a little better here and connect it to the distributional effects ``story'' earlier in the paper}



\section{Appendix}
Painful proofs go here.

\bibliography{references}

\end{document}